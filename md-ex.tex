\documentclass{article}
\usepackage[utf8]{inputenc}
\usepackage[english]{babel}
\usepackage[left=1in,top=1in]{geometry} 
\usepackage{minted}
 
\begin{document}
\begin{minted}{md}
---
layout: node
title: Shell Model
uuid: shell-model
prerequisites:
  - periodic-table
learning_objectives: 
  - reproduce a shell model of an atom 
references:
  - None
abet_outcomes: None
assessments: 
  - shell-model.yml 
...

### Overview
The most straightforward concept of the atom is the shell model, first proposed 
by Niels Bohr.

A nucleus contains protons and neutrons. Protons carry a positive charge, and 
neutrons carry no charge. The nucleus is surrounded by shells of 
negatively-charged electrons.

Each shell can only hold a fixed number of electrons, and each shell 
essentially represents a principal energy level. The electrons orbit around the 
nucleus. 

(Quantum physics has shown this is more of an electron cloud, and there is a 
limit to how precise one can simultaneously know the position or momentum of a 
particle; aka the Heisenberg Uncertainty Principle. For now though, we are only 
concerned with the Bohr shell model.)

### Example
The calcuim atom contains 20 protons and 20 neutrons.

![Ca shell atom](../img/calcium.gif)

The uranium atom contains 92 protons, the number of neutrons will be different 
if the atom is $$^{235}U$$ or $$^{238}U$$.

![U shell atom](../img/uranium.jpg)

Each electron shell is label by its principal quantum number; e.g., 1, 2, 3, 4, 
etc., with the lower number closer to the nucleus.
    
The [dynamic periodic table](https://ptable.com/) gives a lot of information 
about all the elements.
\end{minted}
\end{document}
